
Even if you are not using \ocpbuild{} to build you own projects, you
might need some more information to take advantage of \ocpbuild{}
features when compiling other projects.

\section{Invoking \ocpbuild{}}

If you want to compile a project that uses \ocpbuild{}, you can use
the following commands:

\begin{description}
\item[{\tt ocp-build init}:] This command defines the root directory
  of the project as the current directory. \ocpbuild{} will create a
  directory {\tt \_obuild}. When invoked from any sub-directory, it
  will come back to this directory to compile the full project.
%\item[{\tt ocp-build configure [OPTIONS]}:] This command can be used
%  to modify default options of the project, stored in the {\tt
%    ocp-build.root} file. 
%\item[{\tt ocp-build project [QUERY]}:] This command can be used to
%  read the project, and query some informations. For example, it can
%  be used to ask which packages are going to be built in the current
%  configuraiton.
\item[{\tt ocp-build build [PACKAGES]}:] This command can be used to
  compile all the project, or just a list of packages.
\item[{\tt ocp-build install [PACKAGES]}:] install the specified
  packages.
\item[{\tt ocp-build test}:] run the project tests
\item[{\tt ocp-build clean}:] remove the {\tt \_obuild} directory
  containing all build artefacts.
\item[{\tt ocp-build query [QUERY]}:] This command can be used to
  query some information from the OCaml environment.
\end{description}


\section{The {\tt \_obuild} directory}

\ocpbuild{} creates a directory {\tt \_obuild} at the root of the
project, to store the files it creates. The content of the {\tt
  \_obuild} directory is as follows:
\begin{description}
\item[{\tt \_obuild/cache.cmd}:] a list of command checksums, to avoid
  re-executing commands already executed.
\item[{\tt \_obuild/cache.cmd.log}:] the log of texts used to compute
  command checksums.
\item[{\tt \_obuild/\_mutable\_tree/}:] a copy of the project hierarchy
  of directories, used to store temporary sources and annotations
  (actually, redirections towards annotation files)
\item[{\tt \_obuild/}package/:] for each package, a directory
  containing the object files generated during the compilation.
\end{description}


\section{Setting \ocpbuild{} default parameters}
